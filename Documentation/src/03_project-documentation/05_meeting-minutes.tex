\chapter{Meeting Minutes}

\section{Kickoff Meeting 22.02.2022}

\subsection{Goals}
\begin{itemize}
    \item Choose tools (issue management, time tracking, VCS)
    \item Define and assign roles
    \item Rough long term plan
    \item Finalize vision of project
    \item Define and plan basic meetings
\end{itemize}

\subsection{Vision Discussion}

After playing a "Jass" while using the existing Excel worksheet we found the following new features which might be implemented:

\begin{itemize}
    \item Disable input for teams which already played
    \item Add clock to see how long a player has been thinking (maybe not ideal because tracking a "Schiebe" is tedious)
\end{itemize}

\subsection{Results}
\begin{itemize}
    \item We decided to use GitLab for everything
    \item Assign basic scrum roles
    \item Play a "Jass", ensure everybody knows basic rules
    \item Define and plan basic meetings and milestones
\end{itemize}

\section{Project Meeting 01.03.2022}

\subsection{Goals}
\begin{itemize}
    \item Incorporate new information from SEP2 lecture
    \item Finalize project plan
\end{itemize}

\subsection{Results}
\begin{itemize}
    \item We decided to ditch GitLab in favor of Jira for project planning after noticing the limited functionality of GitLab for project planning
    \item Setup Jira and create rough project plan using epics
\end{itemize}

\section{Project Meeting 05.03.2022}

\subsection{Goals}
\begin{itemize}
    \item Risk management
    \item Hand in documents for Review Meeting 1
    \item First sprint planing for next 2 weeks
\end{itemize}

\subsection{Results}
\begin{itemize}
    \item Risk assessment done
    \item Sprint 2 in Jira planned
    \item First version of documentation provided to project reviewer
\end{itemize}

\section{Review Meeting 1 07.03.2022}

\subsection{Feedback}
Overall on track, no major problems but some smaller ones.

\begin{itemize}
    \item Chapter 1 (Vision) noch leer
    \begin{itemize}
        \item Sollte ähnlichen Inhalt wie "Project Proposal" enthalten
        \item Wird typischerweise als Teil der "Inception" erstellt
    \end{itemize}
    \item Chapter 5 (Quality Measure) auf SEHR gutem Weg \textrightarrow Bravo!
    \item Tipp: Verantwortlichkeiten von Rollen präzisieren / festhalten
    \begin{itemize}
        \item Was macht ein QA? Macht der PO "alle" RE-Tätigkeiten? Nur Priorisierung? etc.
    \end{itemize}
    \item Denkanstoss: Braucht es einen PM (Project Manager?)
    \item Tipp: Time-Boxes für Meetings definieren und so Klarheit schaffen
    \item Wo sehe ich in JIRA eine Übersicht aller RUP-Phasen und Milestones?
    \begin{itemize}
        \item Antwort Team: RUP-Phasen gingen im Rahmen der Planung vergessen ;-)
    \end{itemize}
    \item Warum ist "Usability Testing" eine eigene Epic? Teil einer Story? Teil der DoR oder DoD?
    \item Auf welcher Basis / Schätzung wurde die "langfristige" Planung erstellt? \textrightarrow A: Bauchgefühl, OK
    \item Die Planung für Sprint 02 ist cross-functional =\textrightarrow sehr gut!
    \begin{itemize}
        \item Aber: welche \textbf{Funktionalität} ist am Ende von Sprint 2 zu erwarten? \textrightarrow "Sprint Goal"
    \end{itemize}
    \item Selbes Risiko in "Very High" und "High" (Used technologies) \textrightarrow A: Copy \& Paste-Fehler
    \item High - R2: Weitere Massnahme: Priorisierung des Backlogs (wichtiges zuerst)
    \item Welchen Einfluss auf die Planung haben die ermittelten Risiken?
    \begin{itemize}
        \item Vermutung: Teilweise sind diese in Q-Massnahmen eingeflossen
        \item Antwort Team: Pufferzeiten in "Roadmap" einfliessen lassen \textrightarrow sehr gut!
    \end{itemize}
    \item Noch kein Code der gebaut werden kann, aber GitLab-Repository vorbereitet
    \begin{itemize}
        \item Wird bei R2 nochmals geprüft
    \end{itemize}
    \item Wo finde ich einen Gesamtüberblick über die Zeitreports?
    \begin{itemize}
        \item Fehlend: bisher verbraucht / verbleibend? Abweichung IST von SOLL? etc.
        \item Antwort Team: in JIRA möglich, aber noch nicht angedacht / umgesetzt
        \item Details in JIRA auf Items: OK
    \end{itemize}
\end{itemize}

\subsection{Actions}

\begin{itemize}
    \item Remove RUP, only use scrum
    \item Update documentation according to feedback
    \item Remove "Usability Testing" epic
    \item Document sprint goals in Jira
    \item Start work on exporting time reporting from jira
    \item Create all sprints until the end of the project
    \item Export image of high level project plan into documentation
\end{itemize}

\section{Review Meeting 2 21.03.2022}

\subsection{Feedback}
Overall on track, good iteration output.

\begin{itemize}
    \item Chapter 1 - Vision ergänzt \textrightarrow gut
    \item Info des Teams: kein RUP-Modell mehr; nur noch Scrum orientiert an RUP-Idee \textrightarrow OK
    \item Erster Snapshot separat geliefert; weitere Auswertungen in Arbeit \textrightarrow OK
    \item Feedback aus R1 integriert \textrightarrow Top!
    \item Epics und User Stories in übersichtlichem Format beschrieben \textrightarrow gut
    \item Redundanz zwischen Bericht und JIRA
    \begin{itemize}
        \item Anregung: Braucht es die Akzeptanzkriterien im Bericht? Genügen diese in JIRA? - Anregung: Konsistenz von Bericht und JIRA (Bezeichnungen der Epics + US)
        \item Tipp: Mapping von Bericht zu JIRA über JASS-XYZ statt FR-XY?
    \end{itemize}
    \item Anregung: UC-Diagramm als erstes Kapitel im Kapitel zwecks Übersicht?
    \item Konsistenz: Warum finde ich im Diagramm andere UC/US als in JIRA / im Bericht?
    \item In JIRA \textrightarrow Top!
    \item In JIRA für alle Epics und bald anstehende User Stories \textrightarrow Top!
    \item Nicht nur bald anstehende, sondern alle US mit ersten AC versorgt \textrightarrow sehr gut
    \item Indirekt spezifiziert in 6.2. Processes ("Planning") und Teams-Nachricht
    \begin{itemize}
        \item "Requirements werden fortlaufend detailliert im Jira erarbeitet, unser nächstes Refinement ..."
    \end{itemize}
    \item Sauber strukturiert, schlankes Format \textrightarrow gut
    \item Unter "Measure(s)" stets messbare Kriterien \textrightarrow gut
    \item Keine Priorisierung erkennbar; Vermutung: alle 17 NFR sollen erfüllt werden? Geht das?
    \begin{itemize}
        \item Antwort: Aus Sicht des Teams sollten alle erreichbar sein
    \end{itemize}
    \item Kein Verifikationsprozess definiert: Wer prüft wann und wie die Erfüllung der NFR?
    \begin{itemize}
        \item Antwort: z.T. in JIRA-Stories drin
        \item Verifikation explizit machen (z.B. Zeile "automatisch / manuell" in NFR-Liste)
    \end{itemize}
    \item Schönes DM; kann gut verstanden werden nach Studium der Anforderungen / UI-Skizzen
    \item Zusätzliches Glossary für Mapping der Anforderung / des DM auf GUI-Entwürfe
    \begin{itemize}
        \item Frage / Idee: Was spricht gegen eine Domäne in DE?
    \end{itemize}
    \item Verwendete Begriffe konsistent mit Anforderungen \textrightarrow gut
    \item Auf einem UI-Entwurf gibt es einen "Invite Code"; Anforderung dazu? DM-Konzept? A2 Einfache low-fidelity Entwürfe, helfen sehr für Verständnis des Systems \textrightarrow Top!
    \item Glossary für Mapping von DE nach EN; siehe auch D1
    \item Guter Fortschritt auf vielen Fronten (PP, RE, Code, CI/CD, ...) \textrightarrow sehr gut!
\end{itemize}

\subsection{Actions}

\begin{itemize}
    \item Add column "Automated / Manual" in NFR list
    \item Acceptance Criteria removed from FR, data is managed in JIRA
    \item Linked all FR to associated JIRA issue
    \item moved Use-Case diagram within the document
\end{itemize}

\section{Project Meeting - Retro 22.03.2022}

\subsection{Goals}
\begin{itemize}
    \item What went well (+)
    \item What didn't go well (-)
    \item What can be improved (!)
    \item Sprint Review
\end{itemize}

\subsection{Jamie}
\begin{itemize}
    \item + Generally it worked well
    \item - Time management could have been better, especially knowledge about who has time when to plan work
    \item ! Knowing when everybody is available and not finishing most things on the last weekend
\end{itemize}

\subsection{Marcel}
\begin{itemize}
    \item + Generally everything worked well
    \item - Nothing specific to add
    \item ! Improve time management / planning
\end{itemize}

\subsection{Pascal}
\begin{itemize}
    \item + Good work for the first sprint, everybody now seems to have a better understanding about project.
    \item - Time management, especially with stories regarding documentation / admin work because of the latex takes up a lot of time, not have everything finished on last weekend
    \item ! Estimate more time for documentation / admin stories, complete stories earlier instead of having them nearly finished
\end{itemize}

\subsection{David}
\begin{itemize}
    \item + Good for first sprint, good to see how our estimations were and how we should improve
    \item - Time management between members, some had more stories which lead them to invest more than others, but it should be more equally divided
    \item ! Try and ensure that all members have roughly the same amount of work assigned
\end{itemize}

\subsection{Sprint Review}
\begin{itemize}
    \item We accomplished our sprint goal and completed all required stories
    \item Should try to finish stories earlier instead of completing them on last weekend
    \item Plan more time for documentation / admin stories
\end{itemize}

\section{Project Meeting - Retro 05.04.2022}

\subsection{Goals}
\begin{itemize}
    \item What went well (+)
    \item What didn't go well (-)
    \item What can be improved (!)
    \item Sprint Review
\end{itemize}

\subsection{Jamie}
\begin{itemize}
    \item + Improved skills to ask for help and got help to have a better understanding about project
    \item - Had to ask a lot of questions, thus spent more time on issues
    \item ! Take issues and try self before asking for help, Pair Programming
\end{itemize}

\subsection{Marcel}
\begin{itemize}
    \item + DB setup/migrations weren't painful and went well
    \item - Wasted too much time for routing issue
    \item ! Stop digging further into issue after a certain time spend - prevent wasting time
\end{itemize}

\subsection{Pascal}
\begin{itemize}
    \item + A lot of progress, review meeting, laid a good foundation for further development, enjoyed refactoring everything
    \item - Stories could had been better divided between members
    \item ! Make stories smaller to implement independently
\end{itemize}

\subsection{David}
\begin{itemize}
    \item + Good foundation, made a lot of progress, got far with not investing too may hours
    \item - Not miss easy points for review feedback
    \item ! Plan working on review documentation and not missing easy points
\end{itemize}

\subsection{Sprint Review}
\begin{itemize}
    \item Time estimations were relatively good, good progress was made
    \item Few stories were fully implemented because of time, spreading of stories of sprint
    \item Spread out stories over sprint better, work on issues that are dependent on others first
\end{itemize}

\section{Project Meeting - Retro 19.04.2022}

\subsection{Goals}
\begin{itemize}
    \item What went well (+)
    \item What didn't go well (-)
    \item What can be improved (!)
    \item Sprint Review
\end{itemize}

\subsection{Jamie}
\begin{itemize}
    \item + First Week of Sprint, Quality Tool Story - took it and worked on it instead of waiting for assignment
    \item - Motivation
    \item ! ?
\end{itemize}

\subsection{Marcel}
\begin{itemize}
    \item + Could get to know the codebase better
    \item - Could have done more
    \item ! Prioritization with workload of other modules
\end{itemize}

\subsection{Pascal}
\begin{itemize}
    \item + Got on well, UI got new features, worked without too much pain, could solve problems around
    \item - Invested Time, wasn't fully evenly split up, Risk that we invest too little time for everything we want to do
    \item ! Either commit better to proposed time or change sprint goals
\end{itemize}

\subsection{David}
\begin{itemize}
    \item + First week went pretty well, laid good foundation with backend/frontend communication and state management
    \item - Second week went not as good, plus waiting dependencies 
    \item ! Better planning of issues in sprint itself
\end{itemize}

\subsection{Sprint Review}
\begin{itemize}
    \item Went okay, in frontend not that much progress visible, more was done behind the hood
    \item Sprint goal was not accomplished, thus optimized the sprint goal of next sprint
    \item Maybe implement more things that are visible for user to see progress, better planning of issues in sprint
\end{itemize}
